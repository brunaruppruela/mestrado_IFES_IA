% Created 2023-11-11 sáb 19:51
% Intended LaTeX compiler: lualatex
\documentclass{scrartcl}
\usepackage[margin=2cm,includefoot,footskip=30pt]{geometry}\usepackage{layout}\usepackage{mathtools}
\usepackage{fvextra}
\usepackage{fancyvrb}
\usepackage{amsmath}
\usepackage[dvipsnames,table]{xcolor}
\usepackage{listings}
\usepackage{fixltx2e}
\usepackage{fontspec}
\defaultfontfeatures{Ligatures=TeX,Scale=MatchLowercase}
\setmainfont[SmallCapsFont={Linux Libertine Capitals}]{Linux Libertine O}
\setsansfont[SmallCapsFont={Linux Biolinum Capitals}]{Linux Biolinum O}
\setmonofont{Source Code Pro}
\DefineVerbatimEnvironment{verbatim}{Verbatim}{fontsize=\small,formatcom={\color[rgb]{0.5,0,0}},breaklines=true}
\input{/home/jefferson/Sync/tex/custom-listings}

\usepackage{graphicx}
\usepackage{longtable}
\usepackage{wrapfig}
\usepackage{rotating}
\usepackage[normalem]{ulem}
\usepackage{amsmath}
\usepackage{amssymb}
\usepackage{capt-of}
\usepackage{hyperref}
\usepackage{fvextra}
\fvset{breakbefore=(}
\usepackage{syntax}
\author{Prof. Dr. Jefferson O. Andrade}
\date{2023/2}
\title{Interpretador PL/0 em Clojure\\\medskip
\large Teoria da Computação/Linguagens Formais e Autômatos}
\hypersetup{
 pdfauthor={Prof. Dr. Jefferson O. Andrade},
 pdftitle={Interpretador PL/0 em Clojure},
 pdfkeywords={},
 pdfsubject={},
 pdfcreator={Emacs 28.2 (Org mode 9.6.11)}, 
 pdflang={English}}
\begin{document}

\maketitle


\section{Introdução}
\label{sec:org4b57d08}

Este arquivo descreve a implementação de um interpretador para a \href{https://en.wikipedia.org/wiki/PL/0}{Linguagem PL/0}.
O interpretador descrito aqui considera a versão revisada da linguagem que
inclui primitivas para entrada e saída de dados, representadas pelos símbolos
\textbf{“?”} e \textbf{“!”}, respectivamente.

Além disso, é importante notar que este arquivo está sendo gerado pelo sistemas
\href{https://orgmode.org/}{Org Mode} do editor de textos \href{https://www.gnu.org/software/emacs/}{GNU Emacs} -- ou simplesmente \emph{Emacs} -- e o código
fonte Clojure será gerado extraindo-se os blocos de código deste arquivo\footnote{Depois que for feito o processo de \texttt{tangle}, isso não será muito
importante. Existirá um arquivo chamado \texttt{pl0.clj} com todo o código Clojure nele
e este será o arquivo usado pelo complilador para construir o programa.}.
Ou seja, para facilitar o entendimento do código estamos utilizando uma técnica
chamada \href{https://en.wikipedia.org/wiki/Literate\_programming}{programação literada} (\emph{“literate programming”} em inglês). Por esse
motivo, começamos declarando o \emph{namespace} e as bibliotecas que serão
utilizadas.

\begin{lstlisting}[language=Lisp,label= ,caption= ,captionpos=b,firstnumber=1,numbers=left]
(ns ifes.pl0.core
  (:require [instaparse.core :as insta]
            [clojure.core.match :refer [match]]))
\end{lstlisting}


\section{Gramática da Linguagem PL/0}
\label{sec:org7fe97bb}

Para processar os caracteres (texto) de entrada na linguagem PL/0 será usada a
biblioteca \href{https://github.com/Engelberg/instaparse}{Instaparse}. Esta biblioteca lê a definição da gramática em formato
EBNF e gera uma função que recebe um texto, processa esse texto de acordo com as
definições da gramática e gera o \emph{Abstract Syntax Tree} (AST) correspondente.

Para a linguagem PL/0 foi gerado o arquivo de gramática
\texttt{\$PRJ/resources/pl0.bnf}\footnote{A variável \texttt{\$PRJ} será usada para indicar o diretório raiz do projeto.}, mostrado abaixo:

\begin{lstlisting}[language=ebnf,label=src:pl0.bnf,caption= ,captionpos=b,firstnumber=1,numbers=left]
<program> = block <"."> .

block = [ const_decl ]
        [ var_decl ]
        { proc_decl }
        statement .

const_decl = <"const"> ident "=" number {<","> ident "=" number} <";"> .

var_decl = <"var"> ident {<","> ident} <";"> .

proc_decl = <"procedure"> ident <";"> block <";"> .

statement = [ ident ":=" expression
            | "call" ident 
            | "?" ident
            | "!" expression
            | "begin" statement { <";"> statement } <"end">
            | "if" condition <"then"> statement 
            | "while" condition <"do"> statement
            ].

condition = "odd" expression
          | expression ("="|"<>"|"#"|"<"|"<="|">"|">=") expression .

expression = term { ("+"|"-") term }.

term = factor {("*"|"/") factor}.

factor = ["+"|"-"] factor | ident | number | <"("> expression <")"> .

ident = #"[A-Za-z_][0-9A-Za-z_]*".

number = #"[0-9]+".
\end{lstlisting}

Para poder lidar com espaços em branco e comentários nos arquivos de entrada
PL/0, é necessário definir uma gramática à parte -- e mais tarde um parser
separado. A gramática para espaços em branco e comentários foi definida no
arquivo \texttt{\$PRJ/resources/pl0-ws.bnf} e é dada abaixo.

\begin{lstlisting}[language=ebnf,label=src:pl0ws.bnf,caption= ,captionpos=b,firstnumber=1,numbers=left]
ws-or-comment = (#'\s+' | comment)+ .
<comment> = line-comment | block-comment .
line-comment = <'//'> #'[^\n\r]*' .
block-comment = <'(*'> inside-comment* <'*)'> .
<inside-comment> = !( '*)' | '(*' ) #'.|[\n\r]' | block-comment .
\end{lstlisting}

Consulte a documentação do Instaparse para uma descrição detalhada de como
definir gramáticas.


\section{Definição do Parser da Linguagem PL/0}
\label{sec:org93b4fce}

Uma vez que a gramática esteja definida, podemos usar a função \texttt{parser} do
pacote Instaparse para construir uma função que fará a análise sintática
(\emph{parsing}) dos códigos em PL/0. Entretanto, primeiro precisamos construir o
parser para lidar com espaços em branco e comentários e só depois podemos
construir o parser principal.

\begin{lstlisting}[language=Lisp,label= ,caption= ,captionpos=b,firstnumber=6,numbers=left]
(def whitespace-or-comment
  "Parser que se encarrega de identificar os espaços em branco e comentários do
  código fonte."
  (insta/parser (clojure.java.io/resource "pl0-ws.bnf")))

(def parser
  "Parser que reconhece o código fonte da linguagem PL/0. Note que o parser
  `whitespace-or-comment` é passado para a opção `:auto-whitespace`."
  (insta/parser (clojure.java.io/resource "pl0.bnf")
                :auto-whitespace whitespace-or-comment))
\end{lstlisting}

Note que na criação do analisador sintático principal, \texttt{parser}, foi passada a
opção \texttt{:auto-whitespace whitespace-or-comment}, que instrui o criador do
analisador sintático a usar o analisador sintático secundário
\texttt{whitespace-or-comments} para lidar com espaços em branco.


\section{Estruturas de Dados do Interpretador PL/0}
\label{sec:orgcceb69d}

\subsection{Abstract Syntax Tree}
\label{sec:orgaa821b7}

O \emph{Abstract Syntax Tree} da linguagem PL/0 foi implicitamente definido ao se
definir a gramática da linguagem\footnote{Assim como de qualquer outra definida usando-se o Instaparse.}. Se temos, por exemplo, uma regra de
produção que diz:
\begin{lstlisting}[language=ebnf,label= ,caption= ,captionpos=b,numbers=none]
assign = variable "<-" number ";" .
\end{lstlisting}
E o parser criado à partir dessa gramática recebe a cadeia de entrada \texttt{"x <-
10"}, então a estrutura de dados que será retornada é:
\begin{lstlisting}[language=Lisp,label= ,caption= ,captionpos=b,numbers=none]
[:assign [:variable "x"] "<-" [:number "10"] ";"]
\end{lstlisting}
Como relação a isso, o máximo de controle que se tem é dizer que alguns símbolos
podem ser descartados. Por exemplo, os símbolos \texttt{"<-"} e \texttt{";"} na regra de
produção acima tem função apenas sintática, sem nenhuma importância semântica.
Podem ser descartados sem nenhuma perda de informação. A sinalização de que um
símbolo pode ser descartado é feita através de parênteses angulares.
\begin{lstlisting}[language=ebnf,label= ,caption= ,captionpos=b,numbers=none]
assign = variable <"<-"> number <";"> .
\end{lstlisting}
Nesta nova gramática a estrutura de dados retornada é:
\begin{lstlisting}[language=Lisp,label= ,caption= ,captionpos=b,numbers=none]
[:assign [:variable "x"] [:number "10"]]
\end{lstlisting}


\subsection{Ambiente de Execução}
\label{sec:org28905c0}

Para o interpretador funcionar adequadamente não é suficiente ter o AST
representando o código fonte em PL/0 que foi passado. É necessário ter algum
meio de manter registro do comportamênto dinâmico do programa. Ou seja, de como
o estado do programa evolui com o tempo.

Toda variável, toda constante, toda função ou procedimento que são definidos no
programa PL/0 precisam estar “guardados” em alguma estutura de dados para que o
interpretador tenha acesso a essas definições quando for necessário para
interpretar algum código. Mais ainda, é necessário que exista uma forma de
isolar algumas definições de outras.

Por exemplo, se temos um procedimento \texttt{p1} e outro procedimento \texttt{p2}, não
podemos deixar que \texttt{p1} acesse as var iáveis declaradas em \texttt{p2} e vice-versa.
Mas tanto \texttt{p1} quanto \texttt{p2} precisam ter acesso às variáveis declaradas no
programa principal. De modo mais concreto, considere o código PL/0 abaixo:

\begin{lstlisting}[language=pascal,label= ,caption= ,captionpos=b,firstnumber=1,numbers=left]
var ret, arg, acc;
procedure p1;
var x;
begin
   x := arg;
   ret := x / 3
end;
procedureprocedure p2;
var y;
begin
   y := arg;
   ret := y * 2;
end;
begin
   ?arg;
   acc := 0;
   p1;
   acc := acc + ret;
   p2;
   acc := acc + ret;
   !acc
end.
\end{lstlisting}

É necessário que tanto \texttt{p1} quanto \texttt{p2} tenham acesso a \texttt{arg} e a \texttt{ret}, mas
\texttt{p1} não pode ter acesso a \texttt{y} e \texttt{p2} não pode ter acesso a \texttt{x}. E no program a
principal não deve ser possível acessar nem \texttt{x} e nem \texttt{y}.

A forma mais direta de resolver essa questão é criar um ambiente de execução que
tenha níveis que espelhem os níveis de execução do código. Toda vez que um
procedimento for executado um novo nível no ambiente de execução é criado e o
procedimento será executado neste novo nível que foi criado. Evidentemente, esse
novo nível apontará para o nível que existia no momento em que ele foi criado.
Se for pedido um valor associado a um nome que não existe no nível atual do
ambiente de execução, esse valor será buscado nos níveis superiores, até que se
atinja o nível mais alto.

Para implementar o ambiente de execução dos programas do nosso interpretador,
vamos definir dois registro em Clojure: \texttt{Binding} e \texttt{Env}.

\subsubsection{Binding}
\label{sec:orga015136}

Um \texttt{Binding} (amarração) é a estrutura de dados que está associada a um nome no
ambiente em que o código PL/0 estará sendo executado. Um \texttt{Binding} possui dois
campos:
\begin{itemize}
\item \texttt{:kind} -- o tipo de binding que está sendo criado (:var, :const, :proc)
\item \texttt{:value} -- o valor que está associado ao binding
\end{itemize}


\begin{lstlisting}[language=Lisp,label= ,caption= ,captionpos=b,firstnumber=23,numbers=left]
(defrecord Binding [kind value])
\end{lstlisting}

Ao criar o registro \texttt{Binding} também são implicitamente definidas as funções
\texttt{->Binding} e \texttt{map->Binding} para criação de objetos desse tipo.


\subsubsection{Env}
\label{sec:org51b6d2a}

Um `Env` (environment/ambiente) é a estrutura de dados que irá armazenar
todas as informações necessárias para a execução do código dos programas
PL/0. Um `Env` possui dois campos:
\begin{itemize}
\item \texttt{:bindings} -- guarda todos os nomes definidos no ambiente. Cada nome está
associado a um `Binding`.
\item \texttt{:parent} -- guarda o ambiente pai do ambiente atual. Se este valor for \texttt{nil},
significa que este ambiente é o ambiente raiz.
\end{itemize}


\begin{lstlisting}[language=Lisp,label= ,caption= ,captionpos=b,firstnumber=24,numbers=left]
(defrecord Env [bindings parent])

(defn new-env
  "Cria um novo ambiente. Se não for especificado `parent`, é criado um ambiente
  raiz, caso contrário é criado um ambiente filho do ambiene indicado."
  ([] (->Env {} nil))
  ([parent] (->Env {} parent)))
\end{lstlisting}


Além das funções implicitamente definidas padrão, \texttt{->Env} e \texttt{map->Env}, também
foi criada explicitamente uma função \texttt{new-env} para construção de um novo
ambiente sem nenhum \emph{binding}.

Abaixo são definidas várias funções auxiliares para lidar com os ambientes de
execução.


\begin{lstlisting}[language=Lisp,label= ,caption= ,captionpos=b,firstnumber=31,numbers=left]
(defn is-defined?
  "Retorna `true` se o `name` estiver definido em `env`, e `false` caso contrário.
  Testa o ambiente no nível atuale toda a sequência de `parent` até o nível
  raiz."
  [env name]
  (or (contains? (:bindings env) name)
      (and (some? (:parent env)) (is-defined? (:parent env) name))))


(defn def-name
  "Acrescenta uma definição do tipo (`kind`) indicado para `name` com o `value`
  indicado. Não valida se o valor é apropriado para o tipo informado."
  [env name kind value]
  (assoc-in env [:bindings name] (->Binding kind value)))

(defn def-var [env name] (def-name env name :var nil))
(defn def-const [env name value] (def-name env name :const value))
(defn def-proc [env name body] (def-name env name :proc body))


(defn set-var
  "Retorna um novo ambiente à partir de `env` com o valor `value` associado à
  variável `name`. Gera um erro se `name` não estiver definido ou se não for uma
  variável."
  [env name value]
  ;; (printf "(set-var %s %s %s)\n" env name value)
  (cond
    (some? (get-in env [:bindings name]))
    (let [v (get-in env [:bindings name])]
      (match [(:kind v)]
        [:var] (assoc-in env [:bindings name :value] value)
        [:const] (throw (ex-info (str "Name is a constante: " name) v))
        [:proc] (throw (ex-info (str "Name is a procedure: " name) v))))
    (some? (:parent env)) (assoc env :parent (set-var (:parent env) name value))
    :else (throw (ex-info (str "Undefined variable: " name) {}))))


(defn get-value
  "Retorna o valor da variável ou constante `name` no ambiente `env`. Gera um
  erro se `name` não for definido, ou se não for uma variável ou uma constante."
  [env name]
  ;; (printf "(get-value %s %s)\n" env name)
  (cond
    (some? (get-in env [:bindings name]))
    (let [v (get-in env [:bindings name])]
      (match [(:kind v)]
        [:var] (:value v)
        [:const] (:value v)
        [:proc] (throw (ex-info (str "Name is a procedure: " name) v))))
    (some? (:parent env)) (recur (:parent env) name)
    :else (throw (ex-info (str "Undefined variable or constant: " name) {}))))


(defn get-proc
  "Retorna a definição do procedimento `name` no ambiente `env`. Gera um erro se
  `name` não for definido, ou se não for um procedimento."
  [env name]
  (cond
    (some? (get-in env [:bindings name]))
    (let [v (get-in env [:bindings name])]
      (match [(:kind v)]
        [:var] (throw (ex-info (str "Name is a variable: " name) v))
        [:const] (throw (ex-info (str "Name is a constante: " name) v))
        [:proc] (:value v)))
    (some? (:parent env)) (get-proc (:parent env) name)
    :else (throw (ex-info (str "Procedure not defined: " name) {}))))
\end{lstlisting}


\section{Funções do Interpretador PL/0}
\label{sec:org18d1268}

Serão definidos dois tipos de função:
\begin{enumerate}
\item as funções \texttt{exec} são para execução de declarações e comandos e retornam
ambientes;
\item as funções \texttt{eval} são para avaliação de condições e expresões e retornam o
valor da condição ou expressão.
\end{enumerate}

Como Clojure exige que a função seja definida -- ou ao menos declarada -- antes
de ser usada, faremos a declaração de todas as funções aqui e definiremos as
funções numa abordagem \emph{top-down}.

\begin{lstlisting}[language=Lisp,label= ,caption= ,captionpos=b,firstnumber=97,numbers=left]
(declare exec-block exec-decl exec-const-decl exec-var-decl
         exec-sttmt exec-progn exec-if exec-while
         eval-expr eval-cond eval-rel-expr)
\end{lstlisting}


\subsection{Funções que Executam Declarações e Comandos}
\label{sec:orgd607570}

A função \texttt{exec} é o ponto de entrada \textbf{principal} do interpretador. Ela recebe o
código fonte do programa PL/0 e interpreta o programa. A função exec passa o
código fonte PL/0 ao \texttt{parser}, exctrai o AST, e passa esse AST à função
\texttt{exec-block}.

\begin{lstlisting}[language=Lisp,label= ,caption= ,captionpos=b,firstnumber=100,numbers=left]
(defn exec
  "Avalia o programa `prog` escrito em PL/0."
  [prog & {:keys [return-env] :or {return-env false}}]
  (let [env (-> prog
                (parser)
                (first)
                (exec-block))]
    (if return-env env nil)))
\end{lstlisting}


a função \texttt{exec-block} processa a definição do programa principal, e também dos
procedimentos. Em PL/0 tanto o programa principal, quanto os procedimentos são
declarados como um bloco (\texttt{block}) na gramática. E um bloco é uma sequência de
declarações, opcionais, de constantes, variáveis, procedimentos, e um comando
(\emph{statement}) obritatório ao final. As declarações de constantes, variáveis e
procedimentos apenas preparam o ambiente para a execução do comando ao final do
bloco.

\begin{lstlisting}[language=Lisp,label= ,caption= ,captionpos=b,firstnumber=108,numbers=left]
(defn exec-block
  "Avalia um bloco de código PL/0. Um bloco de código é uma sequência de
  declarações. A avaliação/execução de cada declaração pode gerar mudanças no
  ambiente de execução do programa."
  ([ast] (exec-block ast (new-env)))
  ([ast env]
   (match [ast]
     [[]] env
     [[:block & decls]] (recur decls env)
     [[d1 & ds]] (->> env
                      (exec-decl d1)
                      (recur ds)))))


(defn exec-decl
  "Avalia/executa uma declaração de um bloco. Uma declaração pode ser: (i)
  declaração de constantes; (ii) declaração de variáveis; (iii) definição de
  procedimento; ou (iv) um `statement`."
  [decl env]
  (match [decl]
    ;; declaração de constantes
    [[:const_decl & const-inits]] (exec-const-decl const-inits env)
    ;; declaração de variáveis
    [[:var_decl & var-ids]] (exec-var-decl var-ids env)
    ;; declaração de procedimentos
    [[:proc_decl [:ident name] body]] (def-proc env name body)
    ;; statement
    [[:statement & _]] (exec-sttmt decl env)))


(defn exec-var-decl
  "Atualiza o ambiente `env` com as definições das variáveis dadas em `idents`.
  Cria um novo ambiente, à partir de `env` em que as variáveis comos nomes dados
  em `idents` estão definidas. Retorna o novo ambiente."
  [idents env]
  (match [idents]
    [[]] env
    [[[:ident name] & idents1]]
    (->> (def-var env name)
         (recur idents1))))


(defn exec-const-decl
  "Atualiza o ambiente `env` com as definições das constantes dadas em `inits`.
  Cria um novo ambiente, à partir de `env` em que as constantes comos nomes e
  valores dados em `inits` estão definidas. Retorna o novo ambiente."
  [inits env]
  (match [inits]
    [[]] env
    [[[:ident name] "=" [:number num] & inits1]]
    (->> (Integer/parseInt num)
         (def-const env name)
         (recur inits1))))
\end{lstlisting}


A função \texttt{exec-sttmt}, que executa um comando, deve ser a última a ser chamada
ao se processar um bloco. Ela também é chamada recursivamente ao se processar
vários tipos de comandos, como o comando de sequência (\texttt{begin} \ldots{} \texttt{end}), o
comando condicional (\texttt{if} \ldots{} \texttt{then} \ldots{}) e o comando de repetição (\texttt{while} \ldots{}
\texttt{do} \ldots{}). Existem 7 tipos de comando em PL/0:
\begin{itemize}
\item \textbf{Atribuição:} associa à variável o valor indicado e retorna o novo ambiente
resultante.
\item \textbf{Chamada de procedimento:} cria um novo ambiente descendente do atual, executa
o procedimento com o novo ambiente e retorna o ambiente no nível atual após a
execução.
\item \textbf{Entrada de dados:} Imprime o nome da variável como \emph{prompt}. Lê um valor da
entrada padrão. Atribui o valor à variável e retorna o ambiente resultante.
\item \textbf{Saída de dados:} Calcula o valor da expressão indicada. Escreve o valor na
saída padrão e retorna o ambiente atual.
\item \textbf{Comando de sequência:} Executa os comandos indicados, em sequência. O
primeiro comando é executado com o ambiente indicado. O ambiente retornado
pelo primeiro comando é usado para executar o segundo comando. O ambiente
retornado pelo segundo comando é usado para executar o terceiro comando e
assim por diante. Retorna o ambiente retornado pelo último comando.
\item \textbf{Comando condicional:} Avalia o valor da condição com o ambiente atual. Se o
valor da condição for diferente de zero, executa o comando indicado com o
ambiente atual e retorna o ambiente que o comando retornar. Caso contrário
retorna o ambiente atual.
\item \textbf{Comando de repetição:} Avalia o valor da condição com o ambiente atual. Se o
valor da condição for zero, retorna o ambiente atual. Caso contrário, executa
o comando indicado com o ambiente atual e usa o ambiente que o comando
retornar para repetir o processo recursivamente.
\end{itemize}

\begin{lstlisting}[language=Lisp,label= ,caption= ,captionpos=b,firstnumber=161,numbers=left]
(defn exec-sttmt
  "Executa um comando (statement) PL/0 e retorn o ambiente resultante."
  [sttmt env]
  ;; (printf "(exec-sttmt %s %s)\n" sttmt env)
  (match [sttmt]
    ;; atribuição
    [[:statement [:ident var] ":=" expr]]
    (let [value (eval-expr expr env)]
      (set-var env var value))
    ;; chamada de procedimento
    [[:statement "call" [:ident name]]]
    (-> (get-proc env name)
        (exec-block (new-env env))
        (get :parent))
    ;; Lê um valor para a variável
    [[:statement "?" [:ident name]]]
    (->> (do (printf "%s? " name)
             (flush)
             (read-line))
         (Integer/parseInt)
         (set-var env name))
    ;; Imprime o valor da expressão
    [[:statement "!" expr]]
    (do (->> (eval-expr expr env)
             (println))
        env)
    ;; Comando de sequência
    [[:statement "begin" & sttmt-seq]]
    (exec-progn sttmt-seq env)
    ;; condicional
    [[:statement "if" cond1 sttmt1]]
    (exec-if cond1 sttmt1 env)
    ;; repetição
    [[:statement "while" cond1 sttmt1]]
    (exec-while cond1 sttmt1 env)
    ;;
    ))


(defn exec-progn
  "Executa os comandos em `sttmts` sequencialmente com o ambiente `env`. O
  primeiro comando é executado com `env`. O ambiente retornado pelo primeiro
  comando é usado para executar o segundo comando. O ambiente retornado pelo
  segundo comando é usado para executar o terceiro comando e assim por diante.
  Retorna o ambiente retornado pelo último comando."
  [sttmts env]
  ;; (printf "(exec-progn %s %s)\n" sttmts env)
  (match [sttmts]
    [[]] env
    [[stt1 & stts]]
    (->> (exec-sttmt stt1 env)
         (recur stts))))


(defn exec-if
  "Avalia o valor de `cnd` com o ambiente `env`; se o valor for diferente de zero,
  executa `sttmt` usando `env` e retorna o ambiente resultante. Caso contrário
  retorna `env`."
  [cnd sttmt env]
  (let [cnd-val (eval-cond cnd env)]
    (if (zero? cnd-val)
      env
      (exec-sttmt sttmt env))))


(defn exec-while
  "Avalia `cnd` em `env`; se o valor for zero, retorna `env`; caso contrário,
  executa `sttmt` em `env` e usa o ambiente resultante para repetir o processo
  recursivamente."
  [cnd sttmt env]
  (let [cnd-val (eval-cond cnd env)]
    (if (zero? cnd-val)
      env
      (->> (exec-sttmt sttmt env)
           (recur cnd sttmt)))))
\end{lstlisting}


\subsection{Funções que Avaliam Condições e Expressões}
\label{sec:org1d5c614}

Quando se trata de avaliar expressões, algumas linguagens de programação são tão
“bem comportadas” assim. Em C, por exemplo, existem diversas formas de uma
expressão mudar o ambiente de execução, e.g., \texttt{(x = 5) + y}, \texttt{(x++ + -{}-{}y)}, etc.
Para linguagens desse tipo são necessários mecanismos mais sofisticados de
interpretação. Em PL/0, felizmente, uma expressão não pode mudar o ambiente de
execução. Isso facilita o processo de interpretação.

No nosso interpretador, as funções que avaliam as condições e expressões recebem
a condição ou expressão a avaliar e o ambiente no qual será veita a avaliação, e
retornam o resultado da avaliação. Sempre considerando que o ambiente não muda
durante o processo de avaliação.

A avaliação é realizada de forma recursiva em uma estrutura descendente que
espelha a estrutura da gramática.


\begin{lstlisting}[language=Lisp,label= ,caption= ,captionpos=b,firstnumber=236,numbers=left]
(declare eval-term eval-factor)

(defn eval-cond
  "Avalia a condição `cnd` de acrodo com as definições de `env` e retorna
  verdadeiro (1) ou falso (0). A condição pode ser `odd <expr>` ou `<expr> <op>
  <expr>`, onde `<op>` é um operador relacional."
  [cnd env]
  (match [cnd]
    [[:condition "odd" expr1]]
    (let [value (eval-expr expr1 env)]
      (mod value 2))
    [[:condition expr1 op expr2]]
    (eval-rel-expr expr1 op expr2 env)))


(defn eval-rel-expr
  "Avalia uma expressão relacional com duas subexpressões, `expr1` e `expr2`, e um operador `op`
  relacional. Retorna 1 se a expressão for verdadeira e 0 caso contrário."
  [expr1 op expr2 env]
  (let [val1 (eval-expr expr1 env)
        val2 (eval-expr expr2 env)]
    (match [op]
      ["="] (if (= val1 val2) 1 0)
      [">"] (if (> val1 val2) 1 0)
      ["<"] (if (< val1 val2) 1 0)
      [(:or "<>" "#")] (if (not= val1 val2) 1 0)
      [">="] (if (>= val1 val2) 1 0)
      ["<="] (if (<= val1 val2) 1 0)
      ;;
      )))


(defn eval-expr
  "Avalia a empressão `expr` no ambiente de execução `env`. Retorna o valor da
  expressão."
  ([expr env] (eval-expr expr nil env))
  ([expr acc env]
  (match [expr]
    [[]] acc
    [[:expression t1 & ts]] (let [v1 (eval-term t1 env)]
                              (eval-expr ts v1 env))
    [["+" t1 & ts]] (let [v1 (eval-term t1 env)]
                      (eval-expr ts (+ acc v1) env))
    [["-" t1 & ts]] (let [v1 (eval-term t1 env)]
                      (eval-expr ts (- acc v1) env)))))


(defn eval-term
  "Avalia o termo `term` no ambiente de execução `env`. Retorna o valor do termo."
  ([term env] (eval-term term nil env))
  ([term acc env]
   (match [term]
     [[]] acc
     [[:term f1 & fs]] (let [v1 (eval-factor f1 env)]
                         (eval-term fs v1 env))
     [["*" f1 & fs]] (let [v1 (eval-factor f1 env)]
                       (eval-term fs (* acc v1) env))
     [["/" f1 & fs]] (let [v1 (eval-factor f1 env)]
                       (eval-term fs (quot acc v1) env)))))


(defn eval-factor
  "Avalia o fator `factor` no ambiente de execução `env`. Retorna o valor do
  fator."
  [factor env]
  ;; (printf "(eval-factor %s %s)\n" factor env)
  (match [factor]
    [[:factor "+" f1]] (let [v1 (eval-factor f1 env)] v1)
    [[:factor "-" f1]] (let [v1 (eval-factor f1 env)] (- v1))
    [[:factor [:ident name]]] (get-value env name)
    [[:factor [:number n]]] (Integer/parseInt n)
    [[:factor [:expression & _]]] (eval-expr (nth factor 1) env)
    ;;
    ))
\end{lstlisting}
\end{document}
